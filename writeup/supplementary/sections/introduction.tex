In the main manuscript, we show that an Fe-Al alloy with nominal Al concentration around 18\%at. shows short-range-ordering (SRO) such that small \DOTHREE and \BTWO precipitates form.
This SRO is observed in samples annealed at both 873~K and 523~K.
With the lower annealing temperature the overall phase fraction of \DOTHREE drops slightly, but there is a slight ripening of the precipitates as the number density of larger clusters increases slightly.
For \BTWO the total number density of clusters both large and small increases, although there are an order of magnitude fewer \BTWO clusters than \DOTHREE clusters to begin with.
Can these observations be replicated using atomistic simulations?
If so, this would give a new level of insight into the mechanisms involved.
Since such simulations are much less expensive than physical experiments, if simulations are found to agree with experiment at the two available temperatures, this would give us confidence to scan the entire interpolating temperature regime much more densely to look for trends and regions of transition.

This problem can be attacked directly using coupled Monte Carlo-Molecular Dynamics (MCMD) calculations which allow a system of comparable size to the experimental domain to equilibrate chemically.
Such calculations allow all competing factors -- potential energy, vibrational entropy, and configurational entropy -- to be directly present in the simulation and competing, and if the total free energy favours SRO this will appear naturally.
However, the downside of such powerful simulations is that they are too expensive to perform using first principles calculations with any respectable simulation domain, so we need to restrict ourselves to classical empirical potentials.
To this end, we survey all four available embedded atom method (EAM) and modified EAM (MEAM) potentials publicly available on the NIST potential repository~\cite{nist}.
Using each potential to build a simple approximation of the Fe-Al phase diagram allows us to immediately rule out two potentials.
While the remaining two can already be seen to show little promise, we proceed to use them for cursory MCMD simulations which, unfortunately, show that neither can reproduce the \DOTHREE-heavy SRO that is observed experimentally.
