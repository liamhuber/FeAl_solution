In the main manuscript, we demonstrate that the machine learning (ML) approach can successfully identify short-range-ordering (SRO) in an Fe-Al alloy with nominal Al concentration around 18\%at.
Applying this methodology to APT tips annealed at 873~K and 523~K, we have searched for both \DOTHREE and \BTWO SRO clusters.
Experimentally, there we find no evidence for \DOTHREE clustering beyond that expected to happen by random chance (i.e. ``statistical clustering'') at either temperature.
For \BTWO, although the absolute number of clusters is small, we do see experimental evidence for such SRO at the lower annealing temperature of 523~K.
Can these observations be replicated using atomistic simulations?
Since such simulations are much less expensive than physical experiments, if simulations are found to agree with experiment at the two available temperatures, this would give us confidence to scan the entire interpolating temperature regime much more densely to look for trends and regions of transition.
In the event that simulations do not match experiment, are there still lessons to be learned about the underlying mechanisms that drive SRO in this system?

This problem can be attacked directly using coupled Monte Carlo / Molecular Dynamics (MCMD) calculations which allow a system of comparable size to the experimental domain to equilibrate chemically.
Such calculations allow all factors contributing to the classical free energy -- potential energy, vibrational entropy, and configurational entropy -- to be directly present in the simulation and competing, and if the total free energy favours SRO this will appear naturally.
However, the downside of such powerful simulations is that they are too expensive to perform using first principles calculations with any respectable simulation domain, so we need to restrict ourselves to classical empirical potentials.
To this end, we survey all four available embedded atom method (EAM) and modified EAM (MEAM) potentials publicly available on the NIST potential repository~\cite{nist}.
Using each potential to build a simple approximation of free energies for the relevant Fe-Al phases allows us to immediately rule out two potentials.
For the remaining two we run MCMD calculation and apply a clustering algorithm to look for the presence and distribution of \DOTHREE and \BTWO SRO/precipitation.
None of our potentials are able to reproduce the experimental results, but for one we are able to see \DOTHREE SRO beyond what is expected statistically.
This clustering behaviour is enhanced at the higher annealing temperature, and the full atomic-scale resolution of the simulations allow us to suggest a mechanism for this behaviour and a path to driving such SRO in practice.
