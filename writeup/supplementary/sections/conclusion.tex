In an attempt to corroborate experimental evidence for SRO in Fe-Al, we have performed atomistic simulations using all four classical potentials publicly available on the NIST repository.
Using simplified 0~K phase diagrams with potential energy (and $T>0~\mathrm{K}$ configurational entropy for the BCC solid solution), we showed that two could be ruled out immediately for not being BCC-stable~\cite{farkas2020model, jelinek2012modified}, one~\cite{mendelev2005effect} showed an appropriate tendency towards solid solution but preferred atypical planar- and columnar-Al organization over \DOTHREE, and the last one~\cite{lee2010modified} promisingly showed a small miscibility gap between moderately high Al-concentration solid solution and \DOTHREE -- at least at sufficiently high temperature.

By running full, coupled MCMD calculations to chemically equilibrate 64~$\mathrm{nm}^3$ periodic cubes of Fe-18\%at. Al, we found that the insights gleaned from the simplified phase diagram were impressively accurate.
Using a clustering algorithm to identify contiguous domains matching reference \DOTHREE and \BTWO patterns, we found that the potential of Lee and Lee~\cite{lee2010modified} does indeed show a strong tendency to form many small cluster (<20 atoms) and some moderately sized clusters (40-270 atoms) at the higher temperature of 873~K, and fewer at the lower temperature of 523~K.
While we did not directly observe the lower temperature simulation to form fewer small but more large precipitates, as observed experimentally, such behaviour is consistent with our simplified phase diagram that predicts a significantly wider miscibility gap at lower temperatures -- i.e. the system is more likely to attempt to fully decompose into low-Al-concentration BCC solid solution and \DOTHREE.
The results for the Mendelev potential~\cite{mendelev2005effect} did not show a strong connection with experiment, but our MCMD calculations were in good agreement with our simplified phase diagram as columnar and especially plate-like Al-rich precipitates could be seen after equilibration.

Overall, with the best performing potential experimental results were qualitatively reproduced, and we are able to offer a hint for why the \DOTHREE precipitates were experimentally observed to skew away from small SRO clusters and towards larger clusters with decreasing annealing temperature.
\todo{Anneal the low-T Lee potential much longer to see if it really decomposes?}
Given the wide gulf between direct atomistic simulations where clusters can be built off an explicitly and exactly known topology, and APT+Machine learning where clustering must be more indirectly inferred, it is not clear how to make a quantitative comparison in the clustering behaviour.
\todo{Pass Yue the equilibrated structures and use them as input for APT simulation + ML clustering analysis? Are larger samples needed for that? The MEAM Lee potential is very expensive and larger may not be easy...}
While the existing potential did a good job reproducing qualitative behaviour, the apparent large miscibility gap between solid solution and \DOTHREE at lower temperatures contradicts the experimental observation of high-Al-concentration BCC solution even at 523~K -- i.e. the potential appears to underestimate the solubility of Al in Fe.
In the future, new potentials may perform better in this regard and allow for a quantitative predictions for how SRO cluster size distributions evolve as a function of annealing temperature.
