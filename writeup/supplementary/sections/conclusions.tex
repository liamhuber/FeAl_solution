In an attempt to corroborate experimental evidence for SRO in Fe-Al, we have performed atomistic simulations using all four classical potentials publicly available on the NIST repository.
By examining 0~K potential energies for various phases (including $T>0~\mathrm{K}$ configurational entropy for the BCC Fe-Al solid solution), we showed that two potentials could be ruled out immediately for not being BCC-stable~\cite{farkas2020model, jelinek2012modified}, one~\cite{mendelev2005effect} showed an appropriate tendency towards solid solution but preferred atypical planar- and columnar-Al organization over \DOTHREE, and the last one~\cite{lee2010modified} promisingly showed a small miscibility gap between moderately high Al-concentration solid solution and \DOTHREE -- at least at sufficiently high temperature.

By running full, coupled MCMD calculations to chemically equilibrate 64~$\mathrm{nm}^3$ periodic cubes of Fe-18\%at. Al, we found that the simplified 0~K ``phase diagram'' gave a useful lense for interpreting observed clustering behaviour.
Using an algorithm to identify contiguous domains matching reference \DOTHREE and \BTWO patterns reveals that the potential of Lee and Lee~\cite{lee2010modified} does indeed show a strong tendency to form dozens of small cluster (<20 atoms) as well as a handful of moderately sized clusters (40-270 atoms) at the higher temperature of 873~K.
At the lower temperature of 523~K the system still wants to decompose into \DOTHREE and almost pure-Fe BCC, as revealed by using an alternative initial configuration for the MCMD calculations.
While we did not directly observe the lower temperature simulation to form fewer small but more large precipitates -- as observed experimentally -- such behaviour is qualitatively consistent with the lower Al solubility at lower temperatures that we do observe.

Overall, with the best performing potential experimental results were qualitatively reproduced, and we are able to offer a hint for why the \DOTHREE precipitates were experimentally observed to skew away from small SRO clusters and towards larger clusters with decreasing annealing temperature -- namely, due to a widening miscibility gap and lower Al solubility which drives fewer, larger clusters.
While the existing potential did a good job reproducing qualitative behaviour, the miscibility gap between solid solution and \DOTHREE at lower temperatures appears to be over-exaggerated in this potential as the experimental observations still show relatively high Al-concentration BCC solution even at 523~K -- i.e. the potential appears to underestimate the solubility of Al in Fe.
In the future, new potentials may perform better in this regard and allow for a quantitative predictions for how SRO cluster size distributions evolve as a function of annealing temperature.
