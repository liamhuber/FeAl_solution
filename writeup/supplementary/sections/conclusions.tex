In an attempt to corroborate experimental evidence for SRO in Fe-Al, we have performed atomistic simulations using all four classical potentials publicly available on the NIST repository.
By examining 0~K potential energies for various phases (including $T>0~\mathrm{K}$ configurational entropy for the BCC Fe-Al solid solution), we showed that two potentials could be ruled out immediately for not being BCC-stable~\cite{farkas2020model, jelinek2012modified}, one~\cite{mendelev2005effect} showed an appropriate tendency towards solid solution but preferred planar- and columnar-Al organization over \DOTHREE, and the last one~\cite{lee2010modified} promisingly showed a small miscibility gap between moderately high Al-concentration solid solution and the expected abutting phase, \DOTHREE -- at least at sufficiently high temperature.

By running full, coupled MCMD calculations to chemically equilibrate 64~$\mathrm{nm}^3$ periodic cubes of Fe-18\%at. Al, we found that the simplified 0~K energies (plus configurational entropy) provide a useful lens for interpreting observed clustering behaviour.
Using an algorithm to identify contiguous domains matching reference \DOTHREE and \BTWO patterns reveals that the potential of Lee and Lee~\cite{lee2010modified} does indeed show a strong tendency to form dozens of small cluster (<20 atoms) as well as a handful of moderately sized clusters (40-270 atoms) at the higher temperature of 873~K.
By using an alternative initial condition (perfect phase decomposition) and by studying the turnover of participation in \DOTHREE clusters, we showed that at the lower temperature of 523~K the system is driven towards perfect phase decomposition of \DOTHREE and almost pure-Fe BCC, with slow consistent growth of precipitates in the MCMD calculations.
In contrast, the situation at 873~K can best be understood to lie above the nose of a TTT diagram;
at this temperature the BCC solution retains a moderate amount of Al and clusters of various sizes readily form, dissolve, and reform later in a different location.
Under no circumstances did our simulations give strong evidence of \BTWO SRO.
%The Mendelev potential does give rod- and plate-like Al organization that is \BTWO-like one- and two-dimensionally.

While we did not directly observe the formation of \BTWO SRO seen in experiment, these calculations do offer hints at the driving mechanisms for the experimentally observed behaviour.
In our calculations, which are not constrained by diffusion kinetics, the lower annealing temperature has a tendency to allow Ostwald ripening of precipitates, while the higher temperature is entropically driven to maintain more and smaller clusters.
These are equivalent to sitting inside and on top of the nose in a TTT diagram, respectively.
In contrast, the experimental observation that SRO occurs at lower but not higher annealing temperatures suggests that perhaps the energetic driving force for forming these clusters is relatively small and easily overcome by the entropic benefit of disorder at the higher temperature, and that kinetic constraints may be preventing these fragile SRO clusters from ripening (i.e. we sit beneath the TTT nose).
We do not have a conclusive explanation for why it is \DOTHREE ordering observed in our calculations while \BTWO ordering is observed experimentally, but it is always possible with empirical potentials that the simulations simply do not represent the interatomic interactions with sufficient fidelity.
We can, however, suggest that \DOTHREE SRO may be obtainable by increasing the nominal Al concentration (so there is more likely to be sufficient Al for local \DOTHREE regimes to form) and increasing the annealing temperature (to prevent such clusters from ripening).