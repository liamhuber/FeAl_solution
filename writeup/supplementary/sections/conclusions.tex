In an attempt to corroborate experimental evidence for SRO in Fe-Al, we have performed atomistic simulations using all four classical potentials publicly available on the NIST repository.
By examining 0~K potential energies for various phases (including $T>0~\mathrm{K}$ configurational entropy for the BCC Fe-Al solid solution), we showed that two potentials could be ruled out immediately for not being BCC-stable~\cite{farkas2020model, jelinek2012modified}, one~\cite{mendelev2005effect} showed an appropriate tendency towards solid solution but preferred planar- and columnar-Al organization over \DOTHREE, and the last one~\cite{lee2010modified} promisingly showed a small miscibility gap between moderately high Al-concentration solid solution and \DOTHREE -- at least at sufficiently high temperature.

By running full, coupled MCMD calculations to chemically equilibrate 64~$\mathrm{nm}^3$ periodic cubes of Fe-18\%at. Al, we found that the simplified 0~K ``phase diagram'' gave a useful lens for interpreting observed clustering behaviour.
Using an algorithm to identify contiguous domains matching reference \DOTHREE and \BTWO patterns reveals that the potential of Lee and Lee~\cite{lee2010modified} does indeed show a strong tendency to form dozens of small cluster (<20 atoms) as well as a handful of moderately sized clusters (40-270 atoms) at the higher temperature of 873~K.
By using an alternative initial condition (perfect phase decomposition) and by studying the turnover of participation in \DOTHREE clusters, we showed that at the lower temperature of 523~K the system is driven towards perfect phase decomposition of \DOTHREE and almost pure-Fe BCC, with slow consistent growth of precipitates in the MCMD calculations.
In contrast, the situation at 873~K can best be understood to lie above the nose of a TTT diagram;
at this temperature the BCC solution retains a moderate amount of Al and clusters of various sizes readily form, dissolve, and reform later in a different location.

While we did not directly observe the lower temperature simulation to form fewer small but more large precipitates -- as observed experimentally -- such behaviour is qualitatively consistent with the picture laid out here: the lower temperature has a tendency to allow Ostwald ripening of precipitates, while the higher temperature is entropically driven to maintain more and smaller clusters.
This effect is much more pronounced in our simulations, perhaps because ripening at the lower temperature in the experiments is kinetically limited, and/or because the empirical potential we used under-estimates the solubility of Al in BCC Fe and thus unfairly penalizes the condition where Al in the matrix can easily come  together to form SRO clusters with a finite lifetime.
